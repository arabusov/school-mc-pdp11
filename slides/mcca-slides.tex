%%% MCCA stands for "Master-class of the Computer Archaeology"
\documentclass[t,aspectratio=169]{beamer}
\usepackage[utf8]{inputenc}
\usepackage[english,russian]{babel}
\usepackage{amsmath}
\usefonttheme[onlymath]{serif}
\usepackage[absolute,overlay]{textpos}
\textblockorigin{10mm}{10mm} % start everything near the top-left corner

\usepackage{verbatim}
\usepackage{listings}

\addtobeamertemplate{navigation symbols}{}{%
    \usebeamerfont{footline}%
    \usebeamercolor[fg]{footline}%
    \hspace{1em}%
    \insertframenumber/\inserttotalframenumber
}

\newcommand{\backupbegin}{
   \newcounter{finalframe}
   \setcounter{finalframe}{\value{framenumber}}
}
\newcommand{\backupend}{
   \setcounter{framenumber}{\value{finalframe}}
}
\usepackage{url}

\author{Андрей Рабусов}
\date{Ноябрь 2022}
\title{Археология компьютеров\\
Мастер-класс}

\begin{document}
\begin{frame}
    \frametitle{Содержание}
    \tableofcontents[sectionstyle=show,subsectionstyle=show]
\end{frame}

\section*{Введение}
\begin{frame}
    \frametitle{Компьютерные технологии сейчас и тогда}
\end{frame}

\section{<<Железо>>}
\begin{frame}
    \frametitle{Двоичные числа}
\end{frame}

\begin{frame}
    \frametitle{Логические операции}
\end{frame}

\begin{frame}
    \frametitle{Переключатели: кирпичи в компьютерах}
\end{frame}

\begin{frame}
    \frametitle{Как запоминает компьютер?}
\end{frame}

\section{Взаимодействие человек--машина}
\begin{frame}
    \frametitle{Перфокарты}
\end{frame}

\begin{frame}
    \frametitle{Перфоленты}
\end{frame}

\begin{frame}
    \frametitle{Телетайп}
\end{frame}

\begin{frame}
    \frametitle{Терминал}
\end{frame}

\section{Программы}
\begin{frame}
    \frametitle{Игры: Space Travel}
\end{frame}

\begin{frame}
    \frametitle{Как обмануть начальство?}
\end{frame}

\begin{frame}
    \frametitle{Что такое программа?}
\end{frame}

\begin{frame}
    \frametitle{UNIX room}
\end{frame}

\begin{frame}
    \frametitle{Командная строка}
\end{frame}

\begin{frame}
    \frametitle{Текстовый редактор}
\end{frame}

\end{document}
